\documentclass[12pt]{exam}

\newcommand{\docname}{Primitive roots activity}

% \printanswers
\lhead[Math 404/504]{Page \thepage\ of \numpages}
\rhead{\docname}
\cfoot{}
\headrule
\usepackage{mathpazo,amsmath,amssymb}
\newcommand{\set}[1]{\left\{\,#1\,\right\}}
\newcommand{\N}{\mathbb N}
\newcommand{\Z}{\mathbb Z}
\newcommand{\Q}{\mathbb Q}
\newcommand{\R}{\mathbb R}
\renewcommand{\labelenumi}{(\alph{enumi})}
\renewcommand{\labelitemi}{$\circ$}
\usepackage{tikz}

\usepackage{listings}
\lstset{basicstyle=\small,frame=l,xleftmargin=.5in}

\begin{document}
\begin{questions}
  \question An \emph{$n$th root of unity} is a solution to $x^n\equiv1\pmod{p}$. We can use our \texttt{powertable} function to find $n$th roots of unity for a given modulus $p$ by inspecting the $n$th column for $1$'s.
  \begin{parts}
    \item Let $p=13$ and find all $n$th roots of unity for $n=2,3,4,5,12$. 
    \item Let $p=17$ and find all $n$th roots of unity for $n=2,3,4,5,16$.
    \item What observations do you make? In each case how many roots are there?
  \end{parts}
  \newpage
  \question An element is called \emph{primitive $n$th root of unity} if it is an $n$th root of unity  and its powers include all the other $n$th roots of unity. Show in each case above that if there is any $n$th root then there is a primitive $n$th root.
  \begin{parts}
    \item $p=13$ and $n=2,3,4,5,12$
    \item $p=17$ and $n=2,3,4,5,16$
  \end{parts}
  \newpage
  \question 
\end{questions}
\end{document}


