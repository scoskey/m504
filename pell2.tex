\documentclass[12pt]{exam}

\newcommand{\docname}{Continued factions and Pell equations activity}

% \printanswers
\lhead[Math 404/504]{Page \thepage\ of \numpages}
\rhead{\docname}
\cfoot{}
\headrule
\usepackage{mathpazo,amsmath,amssymb}
\newcommand{\set}[1]{\left\{\,#1\,\right\}}
\newcommand{\N}{\mathbb N}
\newcommand{\Z}{\mathbb Z}
\newcommand{\Q}{\mathbb Q}
\newcommand{\R}{\mathbb R}
\renewcommand{\labelenumi}{(\alph{enumi})}
\renewcommand{\labelitemi}{$\circ$}
\usepackage{tikz}

\usepackage{listings}
\lstset{basicstyle=\small,frame=l,xleftmargin=.5in}

\begin{document}
\begin{questions}
  \question We have shown Pell equations $x^2-Dy^2=1$ admit solutions. The proof used Dirichlet's approximation theorem applied to $\sqrt{D}$. We now know continued fractions may also be used to approximate $\sqrt{D}$, so we now investigate whether they can help solve Pell equations too.
  
  Find and record the continued fraction of each of the square roots below. They will be periodic, so stop when it repeats. Label the initial portion and the periodic portion.
  \begin{itemize}\itemsep.5cm
    \item $\sqrt{2}$
    \item $\sqrt{3}$
    \item $\sqrt{5}$
    \item $\sqrt{6}$
    \item $\sqrt{7}$
    \item $\sqrt{8}$
    \item $\sqrt{10}$
    \item $\sqrt{11}$
    \item $\sqrt{12}$
    \item $\sqrt{13}$
    \item $\sqrt{14}$
    \item $\sqrt{15}$
    \item $\sqrt{17}$
  \end{itemize}
  
  What observations do you make?
  \newpage
  \question Next we will calculate the convergents $x/y$ for $\sqrt{D}$ and plug them into the left-hand side of the Pell equation $x^2-Dy^2$.
  
  For each of the number below use our \texttt{pell2} to find the first solution to $x^2-Dy^2=\pm1$. In each case how long does it take for the solution to occur?

  The \texttt{pell2(D,B)} function takes the value of $D$ and the number of convergents to try $B$.
  \begin{itemize}\itemsep.5cm
    \item $\sqrt{2}$
    \item $\sqrt{3}$
    \item $\sqrt{5}$
    \item $\sqrt{6}$
    \item $\sqrt{7}$
    \item $\sqrt{8}$
    \item $\sqrt{10}$
    \item $\sqrt{11}$
    \item $\sqrt{12}$
    \item $\sqrt{13}$
    \item $\sqrt{14}$
    \item $\sqrt{15}$
    \item $\sqrt{17}$
  \end{itemize}
  
  What observations do you make? Compare with the data on the previous page.
  \newpage
  \question On the previous page we saw that if $\sqrt{D}=[a_0,\overline{a_1,\ldots,a_k}]$ then the $k$th convergent will always yield a solution to $x^2-Dy^2=\pm1$.
  
  If $x^2-Dy^2=1$ then we have found a solution to the Pell equation.
  
  If on the other hand $x^2-Dy^2=-1$, show that $(x+y\sqrt{D})^2$ gives a solution to $x^2-Dy^2=1$.
\end{questions}
\end{document}
