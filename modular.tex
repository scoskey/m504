\documentclass[12pt]{exam}

\newcommand{\docname}{Modular arithmetic activity}

% \printanswers
\lhead[Math 404/504]{Page \thepage\ of \numpages}
\rhead{\docname}
\cfoot{}
\headrule
\usepackage{mathpazo,amsmath,amssymb}
\newcommand{\set}[1]{\left\{\,#1\,\right\}}
\newcommand{\N}{\mathbb N}
\newcommand{\Z}{\mathbb Z}
\newcommand{\Q}{\mathbb Q}
\newcommand{\R}{\mathbb R}
\renewcommand{\labelenumi}{(\alph{enumi})}
\renewcommand{\labelitemi}{$\circ$}
\usepackage{tikz}

\usepackage{listings}
\lstset{basicstyle=\small,frame=l,xleftmargin=.5in}

\begin{document}
\begin{questions}
  \question We have seen that the addition operation makes sense ``modulo $n$''. This means it is possible to make complete addition tables for each $n$. Make addition tables modulo $11$ and modulo $12$. What patterns do you observe?
  \begin{center}
    \begin{tikzpicture}[scale=.75]
      \draw (0,0) -- (0,12.9);
      \draw (-.9,12) -- (12,12);
      \node at (-.5,12.5) {$+_{11}$};
    \end{tikzpicture}

    \begin{tikzpicture}[scale=.75]
      \draw (0,0) -- (0,12.9);
      \draw (-.9,12) -- (12,12);
      \node at (-.5,12.5) {$+_{12}$};
    \end{tikzpicture}
  \end{center}
  \newpage
  \question We have seen that the multiplication operation makes sense modulo $n$. This means it is possible to make complete multiplication tables for each $n$. Make multiplication tables modulo $11$ and modulo $12$. What patterns do you observe?
  \begin{center}
    \begin{tikzpicture}[scale=.75]
      \draw (0,0) -- (0,12.9);
      \draw (-.9,12) -- (12,12);
      \node at (-.5,12.5) {$\times_{11}$};
    \end{tikzpicture}

    \begin{tikzpicture}[scale=.75]
      \draw (0,0) -- (0,12.9);
      \draw (-.9,12) -- (12,12);
      \node at (-.5,12.5) {$\times_{12}$};
    \end{tikzpicture}
  \end{center}
  \newpage
  \question What about reciprocals and division? Use your multiplication tables to make reciprocal tables modulo $11$ and modulo $12$. If no reciprocal exists, write None. What patterns do you observe?
  \begin{center}
    \begin{tikzpicture}[]
      \draw (0,0) -- (0,12.9);
      \draw (-.9,12) -- (2.9,12);
      \node at (-.5,12.5) {$a$};
      \node at (1.5,12.5) {$a^{-1}\pmod{11}$};
    \end{tikzpicture}
    \quad\quad
    \begin{tikzpicture}[]
      \draw (0,0) -- (0,12.9);
      \draw (-.9,12) -- (2.9,12);
      \node at (-.5,12.5) {$a$};
      \node at (1.5,12.5) {$a^{-1}\pmod{12}$};
    \end{tikzpicture}
  \end{center}
  \newpage
  \question If $\gcd(a,n)=1$, then it is possible to find a reciprocal $a^{-1}$ modulo $n$ by solving the equation $ax+ny=1$ for $x$. For each of the following $a,n$, find $a^{-1}$ modulo $n$, or state none exists.
  \begin{parts}
    \part $a=135$, $n=2342$
    \vspace\fill
    \part 
    \vspace\fill
    \part 
    \vspace\fill
  \end{parts}
  \question If $\gcd(a,n)=1$, then it is possible to solve any equation of the form $ax\equiv b\pmod n$ using reciprocals. Solve each of the following congruence equations using reciprocals.
  \begin{parts}
    \part 
    \vspace\fill
    \part 
    \vspace\fill
    \part 
    \vspace\fill
  \end{parts}
  \newpage
  \question If $\gcd(a,n)=d>1$, it is still possible to solve $ax\equiv b\pmod n$ so long as $d\mid b$. Solve each of the following congruence equations by first dividing both sides by $d$, and then using reciprocals.
  \begin{parts}
    \part 
    \vspace\fill
    \part 
    \vspace\fill
    \part 
    \vspace\fill
  \end{parts}
  \question Each congruence equation from the previous question has more than one solution. Can you find additional solutions?
    \vspace\fill
\end{questions}
\end{document}


