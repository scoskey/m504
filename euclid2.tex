\documentclass[12pt]{exam}

\newcommand{\docname}{Diophantine equations activity}

% \printanswers
\lhead[Math 404/504]{Page \thepage\ of \numpages}
\rhead{\docname}
\cfoot{}
\headrule
\usepackage{mathpazo,amsmath,amssymb}
\newcommand{\set}[1]{\left\{\,#1\,\right\}}
\newcommand{\N}{\mathbb N}
\newcommand{\Z}{\mathbb Z}
\newcommand{\Q}{\mathbb Q}
\newcommand{\R}{\mathbb R}
\renewcommand{\labelenumi}{(\alph{enumi})}
\renewcommand{\labelitemi}{$\circ$}
\usepackage{tikz}

\usepackage{listings}
\lstset{basicstyle=\small,frame=l,xleftmargin=.5in}

\begin{document}
\begin{questions}
  \question Here we adapt the gcd code to return not just the gcd, but also the coefficents $x,y$ such that $d=(x)b+(y)a$.
  \begin{lstlisting}
  def gcd2(b,a):
    a,b = min(a,b), max(a,b)
    if a==0 or b==0:
      print ( "so gcd =", b, "and...")
      return b,1,0
    q, r = b//a, b%a
    print ( b, "=", q, "*", a, "+", r );
    d,x,y = gcd2(a, r)
    x,y = y,x-y*q
    print ( d,"=","(",x,")",b,"+","(",y,")",a )
    return d,x,y
  \end{lstlisting}
  Copy this code to jupyter.org, or if your group agrees, use your own programming setup. Then experiment by plugging many different pairs $a,b$ into the gcd function. What patterns do you observe in the output?
  
  Be sure to consider the sequences of remainders, the length of the computations, and the final answers.
  \newpage
  \question 
\end{questions}
\end{document}


