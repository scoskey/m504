\documentclass[12pt]{exam}

\newcommand{\docname}{Diophantine equations activity}

% \printanswers
\lhead[Math 404/504]{Page \thepage\ of \numpages}
\rhead{\docname}
\cfoot{}
\headrule
\usepackage{mathpazo,amsmath,amssymb}
\newcommand{\set}[1]{\left\{\,#1\,\right\}}
\newcommand{\N}{\mathbb N}
\newcommand{\Z}{\mathbb Z}
\newcommand{\Q}{\mathbb Q}
\newcommand{\R}{\mathbb R}
\renewcommand{\labelenumi}{(\alph{enumi})}
\renewcommand{\labelitemi}{$\circ$}
\usepackage{tikz}

\usepackage{listings}
\lstset{basicstyle=\small,frame=l,xleftmargin=.5in}

\begin{document}
\begin{questions}
  \question The following is a solution to Silverman, exercise 6.3(b).
  \begin{lstlisting}
    def gcd2(a,b):
      a,b = max(a,b), min(a,b)
      if a==0 or b==0:
        print ( f'so gcd = {a:d} and...' )
        return a,1,0
      q, r = a//b, a%b
      print ( f'{a} = ({q}){b} + {r}' )
      d,x,y = gcd2(b, r)
      x,y = y,x-y*q
      print ( f'{d} = ({x}){a} + ({y}){b}' )
      return d,x,y
  \end{lstlisting}
  Copy this code to jupyter.org, or if your group agrees, use your own programming setup.
  \begin{parts}
    \part Use the code to find integer solutions to $ax+by=d$ for the following pairs $a,b$:
    
    \part Find several other solutions to the problems from part (a). Make sure one of your solutions has $x>0$.

    \part Modify the code so that it always returns a solution with $x>0$.
  \end{parts}
  \newpage
  \question Now suppose there are three variables $x,y,z$, and three coefficents $a,b,c$ with no common factors.
  \begin{parts}
    \part Consider the example
    \part Explain how you can use the Euclidean algorithm several times to find integer solutions for $ax+by+cz=1$.
  \end{parts}
\end{questions}
\end{document}


