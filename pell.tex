\documentclass[12pt]{exam}

\newcommand{\docname}{Pell equations activity}

% \printanswers
\lhead[Math 404/504]{Page \thepage\ of \numpages}
\rhead{\docname}
\cfoot{}
\headrule
\usepackage{mathpazo,amsmath,amssymb}
\newcommand{\set}[1]{\left\{\,#1\,\right\}}
\newcommand{\N}{\mathbb N}
\newcommand{\Z}{\mathbb Z}
\newcommand{\Q}{\mathbb Q}
\newcommand{\R}{\mathbb R}
\renewcommand{\labelenumi}{(\alph{enumi})}
\renewcommand{\labelitemi}{$\circ$}
\usepackage{tikz}

\usepackage{listings}
\lstset{basicstyle=\small,frame=l,xleftmargin=.5in}

\begin{document}
\begin{questions}
  \question We have previously studied $x^2-2y^2=1$ and found that $(3,2)$ is the smallest solution and all other solutions may be found using $(x_n,y_n)=(3+2\sqrt{2})^n$.
  
  We now study \emph{Pell equations}: If $D$ be any positive integer that isn't a square, then the following is the Pell equation for $D$:
  \[x^2-Dy^2=1
  \]
  \begin{parts}
    \part Show that if $(x_1,y_1)$ is a solution to Pell's equation with a given $D$, then so is $(x_2,y_2)$, where $x_2+y_2\sqrt{D}=(x_1+y_1\sqrt{D})^2$.
    \vspace\fill
    \part The previous problem helps generate many solutions to a Pell equation from one, but doesn't explain how to find a solution in the first place. In order to begin searching, we examine some properties of solutions.
  
    Show that if $(x,y)$ is a solution to the Pell equation $x^2-Dy^2=1$, then it is true that:
    \[|x-y\sqrt{D}|<\frac{1}{y}
    \]
    (This means $\frac{x}{y}$ is a very close rational approximation to the irrational number $\sqrt{D}$.)
    \vspace\fill
  \end{parts}
  \newpage
  \question The previous page suggested that in order to solve Pell equations we should look for good rational approximations to $\sqrt{D}$. We will use \textbf{Dirichlet's approximation theorem} which states that if $\alpha$ is irrational, there exist infinitely solutions $(x,y)$ to the inequality:
  \[|x-y\alpha|<\frac{1}{y}
  \]
  
  Note that it is easy to find solutions to the inequality $|x-y\alpha|<1$. We can let $y$ be any natural number and then pick the natural number $x$ which is closest to $y\alpha$. Among these solutions, some may happen to be solutions to $|x-y\alpha|<\frac{1}{y}$.
  \begin{parts}
    \item Find at least ten solutions to $|x-y\sqrt{2}|<\frac{1}{y}$. In each case calculate $x/y$.
    
    You may use our \texttt{dirichlet} function to print out solutions to $|x-y\sqrt{2}|<1$ and decide whether they are solutions to $|x-y\sqrt{2}|<\frac{1}{y}$. You will need to use \texttt{from math import sqrt}.
    \vspace\fill
    \item Find at least ten solutions to $|x-y\pi|<\frac{1}{y}$. In each case calculate $x/y$.
    
    Use the \texttt{dirichlet} function again. You will need to use \texttt{from math import pi}.
    \vspace\fill
  \end{parts}
  \newpage
  
\end{questions}
\end{document}
