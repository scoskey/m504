\documentclass[12pt]{exam}

\newcommand{\docname}{Public key cryptography activity}

% \printanswers
\lhead[Math 404/504]{Page \thepage\ of \numpages}
\rhead{\docname}
\cfoot{}
\headrule
\usepackage{mathpazo,amsmath,amssymb}
\newcommand{\set}[1]{\left\{\,#1\,\right\}}
\newcommand{\N}{\mathbb N}
\newcommand{\Z}{\mathbb Z}
\newcommand{\Q}{\mathbb Q}
\newcommand{\R}{\mathbb R}
\renewcommand{\labelenumi}{(\alph{enumi})}
\renewcommand{\labelitemi}{$\circ$}
\usepackage{tikz}

\usepackage{listings}
\lstset{basicstyle=\small,frame=l,xleftmargin=.5in}

\begin{document}
\begin{questions}
  \question Letters of the alphabet may be represented by numbers using the rule `a' $\to$ $11$, `b' $\to$ $12$, \ldots, `z' $\to$ $36$. Strings of letters may be represented by juxtaposing the codes of the letters. For example, `abc' $\to$ $111213$.
  
  Let $(n,k)$ be a public key pair: this means $n$ is a product of two distinct primes and $\gcd(k,\phi(n))=1$. If $m$ is an integer representation of a message and $m<n$, we encrypt $m$ by calculating $m^k\pmod{n}$. If $m\geq n$, one must break $m$ into smaller blocks and encrypt each block separately.
  \begin{parts}
    \part Encrypt the message ``friend'' using $n=4189$ and $k=9$. We must use three blocks of two letters each, so encrypt `fr', `ie', and `nd'. The solution is a list of three elements of $\Z_n$.
    \part 
  \end{parts}
  \newpage
  \question Let $(n,k)$ be the public key pair defined below.
  \begin{parts}
    \part Find $\phi(n)$.
    \part Decrypt the message 
    \part Decrypt the message
  \end{parts}
  \newpage
  \question 
\end{questions}
\end{document}


