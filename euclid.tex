\documentclass[12pt]{exam}

\newcommand{\docname}{Euclidean algorithm activity}

% \printanswers
\lhead[Math 404/504]{Page \thepage\ of \numpages}
\rhead{\docname}
\cfoot{}
\headrule
\usepackage{mathpazo,amsmath,amssymb}
\newcommand{\set}[1]{\left\{\,#1\,\right\}}
\newcommand{\N}{\mathbb N}
\newcommand{\Z}{\mathbb Z}
\newcommand{\Q}{\mathbb Q}
\newcommand{\R}{\mathbb R}
\renewcommand{\labelenumi}{(\alph{enumi})}
\renewcommand{\labelitemi}{$\circ$}
\usepackage{tikz}

\usepackage{listings}
\lstset{basicstyle=\small,frame=l,xleftmargin=.5in}

\begin{document}
\begin{questions}
  \question The following is a plain solution to Silverman, exercise 5.2, using the Python language.
  \begin{lstlisting}
  def gcd(a,b):
    a,b = min(a,b), max(a,b)
    if a==0 or b==0: return b
    q, r = b//a, b%a
    print ( b, "=", q, "*", a, "+", r);
    return gcd(a, r)
  \end{lstlisting}
  Copy this code to jupyter.org, or if your group agrees, use your own programming setup. Then experiment by plugging many different pairs $a,b$ into the gcd function. What patterns do you observe in the output?
  
  Be sure to consider the sequences of remainders, the length of the computations, and the final answers.
  \newpage
  \question The following is Silverman, exercise 5.3.
  \begin{parts}
    \part Explain why in the Euclidean algorithm, the sequence of remainders $r_i$ is strictly decreasing. (This means the Euclidean algorithm always terminates successfully.)
    \vspace{2in}
    \part Show moreover that $r_{i+2}<\frac12 r_i$. (This means the sequence of remainders decays exponentially, and the Euclidean algorithm is efficient.)
    
    [Hint: Consider the cases $r_{i+1}\leq\frac12r_i$ and $r_{i+1}>\frac12r_i$ separately.]
  \end{parts}
  \newpage
  \question In your experimentation, you may have noticed that some gcd computations halt very quickly while others take longer. Recall that the Fibonacci numbers are defined by $F_{n+1}=F_n+f_{n-1}$ and begin:
  \[0,1,1,2,3,5,8,13,21,34,55,89,\ldots
  \]
  \begin{parts}
    \part Experiment with calculating the $\gcd$ of adjacent Fibonacci numbers. What patterns do you observe?
    \vspace\fill
    \part Can you explain why the Euclidean algorithm takes a lot of steps to calculate $\gcd(F_n,F_{n+1})$ compared to other $\gcd$ calculations for comparably sized numbers?
    \vspace\fill
    \part Experiment with calculating the $\gcd$ of non-adjacent Fibonacci numbers. What patterns do you observe?
    \vspace\fill
  \end{parts}
\end{questions}
\end{document}


