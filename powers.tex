\documentclass[12pt]{exam}

\newcommand{\docname}{Modular exponents and roots activity}

% \printanswers
\lhead[Math 404/504]{Page \thepage\ of \numpages}
\rhead{\docname}
\cfoot{}
\headrule
\usepackage{mathpazo,amsmath,amssymb}
\newcommand{\set}[1]{\left\{\,#1\,\right\}}
\newcommand{\N}{\mathbb N}
\newcommand{\Z}{\mathbb Z}
\newcommand{\Q}{\mathbb Q}
\newcommand{\R}{\mathbb R}
\renewcommand{\labelenumi}{(\alph{enumi})}
\renewcommand{\labelitemi}{$\circ$}
\usepackage{tikz}

\usepackage{listings}
\lstset{basicstyle=\small,frame=l,xleftmargin=.5in}

\begin{document}
\begin{questions}
  \question In ordinary arithmetic, powers $a^k$ become incalculably very quickly. In $\Z_n$ we can calculate powers $a^k$ using successive squaring. The following two functions may be used to help with successive squaring.
  
  \begin{lstlisting}
def powermod(a,k,n):
    power, exponent = a, 1
    while exponent<k:
        print( f'{a}^{exponent} = {power} (mod {n})' )
        power, exponent = (power*power) % n, 2*exponent

def binaryexpand(m):
    output, power, q = [], 1, m
    while q > 0:
        q, r = q//2, q%2
        if r == 1: output.append(str(power))
        power = 2 * power
    return(f'{m} = ' + ' + '.join(output))
  \end{lstlisting}
  
  For example to calculate $7^{38}\pmod{53}$, first run \texttt{powermod(7,38,53)} to find that
  \[7^1=7,\quad 7^2=49,\quad 7^4\equiv 16,\quad 7^8\equiv 44,\quad 7^{16}\equiv28,\quad 7^{32}\equiv42\pmod{53}
  \]
  Next run \texttt{binaryexpand(38)} to find that $38=2+4+32$. Finally multiply the corresponding power values \texttt{(49*16*42)\%53} to get the answer $15$.
  \begin{parts}
    \part Find $5^{13}\pmod{23}$
    \vspace\fill
    \part Find $28^{749}\pmod{1147}$
    \vspace\fill
    \part Find $208^{1\,010\,101}\pmod{101}$
    \vspace\fill
  \end{parts}
  \newpage
  \question On the last page we saw how to calculate powers in $\Z_n$ using successive squaring. By inverting the exponent, we can often calculuate roots in $\Z_n$ too. We can find a solution to $x^k\equiv b\pmod{n}$ assuming $\gcd(k,\phi(n))=1$ and $\gcd(b,n)=1$ using the following method.
  \begin{itemize}
    \item Step 1: Calculate $\phi(n)$ using our \texttt{euler} function (or using the \texttt{sympy.totient} function).
    \item Step 2: Find the reciprocal $u=k^{-1}$ in $\Z_{\phi(n)}$ using our \texttt{euclid2} function (or in Python 3.8 using the \texttt{pow} function).
    \item Step 3: Find $b^u\pmod{n}$ using the method of the previous page (or using the \texttt{pow} function).
  \end{itemize}
  \begin{parts}
    \part Solve $x^{329}\equiv452\pmod{1147}$
    \vspace\fill
    \part Solve $x^{113}\equiv347\pmod{463}$
    \vspace\fill
    \part Solve $x^{275}\equiv139\pmod{588}$
    \vspace\fill
  \end{parts}
\end{questions}
\end{document}


