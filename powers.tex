\documentclass[12pt]{exam}

\newcommand{\docname}{Modular exponents and roots activity}

% \printanswers
\lhead[Math 404/504]{Page \thepage\ of \numpages}
\rhead{\docname}
\cfoot{}
\headrule
\usepackage{mathpazo,amsmath,amssymb}
\newcommand{\set}[1]{\left\{\,#1\,\right\}}
\newcommand{\N}{\mathbb N}
\newcommand{\Z}{\mathbb Z}
\newcommand{\Q}{\mathbb Q}
\newcommand{\R}{\mathbb R}
\renewcommand{\labelenumi}{(\alph{enumi})}
\renewcommand{\labelitemi}{$\circ$}
\usepackage{tikz}

\usepackage{listings}
\lstset{basicstyle=\small,frame=l,xleftmargin=.5in}

\begin{document}
\begin{questions}
  \question In ordinary arithmetic, powers $a^k$ become incalculably very quickly. In $\Z_n$ we can calculate powers $a^k$ using successive squaring. The following two functions may be used to help with successive squaring.
  
  \begin{lstlisting}
def powermod(a,k,n):
    power, exponent = a, 1
    while exponent<k:
        print( f'{a}^{exponent} = {power} (mod {n})' )
        power, exponent = (power*power) % n, 2*exponent

def binaryexpand(m):
    output, power, q = [], 1, m
    while q > 0:
        q, r = q//2, q%2
        if r == 1: output.append(str(power))
        power = 2 * power
    return(f'{m} = ' + ' + '.join(output))
  \end{lstlisting}
  
  For example the code
  \begin{parts}
    \part Find
  \end{parts}
  \newpage
  \question On the last page we saw how to calculate powers in $\Z_n$ using successive squaring. By inverting the exponent, we can often calculuate roots in $\Z_n$ too.
  \begin{itemize}
    \item Step 1: Calculate
  \end{itemize}
\end{questions}
\end{document}


