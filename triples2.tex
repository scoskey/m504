\documentclass[12pt]{exam}

\newcommand{\docname}{Math 404/504, Pythagorian triples activity 2}

% \printanswers
\lhead[Name]{Page \thepage\ of \numpages}
\rhead{\docname}
\cfoot{}
\headrule
\usepackage{mathpazo,amsmath,amssymb}
\newcommand{\set}[1]{\left\{\,#1\,\right\}}
\newcommand{\N}{\mathbb N}
\newcommand{\Z}{\mathbb Z}
\newcommand{\Q}{\mathbb Q}
\newcommand{\R}{\mathbb R}
\renewcommand{\labelenumi}{(\alph{enumi})}
\renewcommand{\labelitemi}{$\circ$}
\usepackage{tikz}

\def\multichoose#1#2{\ensuremath{(\kern-.3em(\genfrac{}{}{0pt}{}{#1}{#2})\kern-.3em)}}

\begin{document}
\begin{questions}
  \question The following is just a copy of Silverman's exercise 3.1. Recall from class that give $u,v$ we can generate a Pythagorean triple $(a,b,c)=(u^2-v^2,2uv,u^2+v^2)$.
  \begin{parts}
    \part If $u,v$ have a common factor, show that $(a,b,c)$ will not be a primitive Pythagorean triple.
    \vspace\fill
    \part The converse to (a) is not true! Find an example of integers $u,v>0$ that have no common factor, and the Pythagorean triple $(u^2-v^2,2uv,u^2+v^2)$ is not primitive.
    \vspace\fill
    \newpage
    \part So when is $(a,b,c)$ primitive? Let's try an experimental approach. Make a table of triples $(a,b,c)$ corresponding to various $u,v$.
    \begin{center}
    \begin{tikzpicture}[scale=.75]
      \node at (-1,.5) {${}_v$ \textbackslash\ ${}^u$};
      \draw (-2,0)--(20,0);
      \draw (0,1)--(0,-10);
      \foreach \k in {1,...,10} {
        \node at (2*\k-.5,.5) {$\k$};
        \node at (-1,-\k+.5) {$\k$};
      }
      \node at (3.5,-0.5) {$(3,4,5)$};
    \end{tikzpicture}
    \end{center}
    \part Use the table you made to make a conjecture about what conditions on $u,v$ guarantee that $(a,b,c)$ is a primitive Pythagorean triple. Then try to prove your conjecture is true.
    \vspace\fill
  \end{parts}
\end{questions}
\end{document}


