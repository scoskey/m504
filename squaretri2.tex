\documentclass[12pt]{exam}

\newcommand{\docname}{Square triangles 2 activity}

% \printanswers
\lhead[Math 404/504]{Page \thepage\ of \numpages}
\rhead{\docname}
\cfoot{}
\headrule
\usepackage{mathpazo,amsmath,amssymb}
\newcommand{\set}[1]{\left\{\,#1\,\right\}}
\newcommand{\N}{\mathbb N}
\newcommand{\Z}{\mathbb Z}
\newcommand{\Q}{\mathbb Q}
\newcommand{\R}{\mathbb R}
\renewcommand{\labelenumi}{(\alph{enumi})}
\renewcommand{\labelitemi}{$\circ$}
\usepackage{tikz}

\usepackage{listings}
\lstset{basicstyle=\small,frame=l,xleftmargin=.5in}

\begin{document}
\begin{questions}
  \question We have previously studied \emph{square triangles}, that is, numbers $N$ that simultaneously equal $s^2$ for some $s$ and $t^\triangle=\frac{t(t+1)}{2}$ for some $t$.
  
  We can rearrange the equation $s^2=t^\triangle$ to discover the following equivalence:
  \begin{align*}
    s^2=t^\triangle&\iff s^2=\frac{t(t+1)}{2}\\
    &\iff 8s^2=4t^2+4t\\
    &\iff 2(2s)^2=(2t+1)^2-1\\
    &\iff (2t+1)^2-2(2s)^2=1
  \end{align*}
  In other words, we have found a pair of numbers $x=2t+1,y=2s$ that solve the equation
  \[x^2-2y^2=1
  \]
  The smallest solution to this equation is $(x,y)=(1,0)$. Find the next three solutions to the equation $x^2-2y^2=1$. It may help you to look at the table of squares. (It may help you much more to ``cheat'' and look at our first day activity.)
  \[
    \begin{matrix}
      1&4&9&16&25&36&49&64&81&100\\
      121&144&169&196&225&256&289&324&361&400\\
      441&484&529&576&625&676&729&784&841&900\\
      961&1024&1089&1156&1225&1296&1369&1444&1521&1600\\
      1681&1764&1849&1936&2025&2116&2209&2304&2401&2500\\
      2601&2704&2809&2916&3025&3136&3249&3364&3481&3600\\
      3721&3844&3969&4096&4225&4356&4489&4624&4761&4900\\
      5041&5184&5329&5476&5625&5776&5929&6084&6241&6400\\
      6561&6724&6889&7056&7225&7396&7569&7744&7921&8100\\
      8281&8464&8649&8836&9025&9216&9409&9604&9801&10000
    \end{matrix}
  \]
  \newpage
  \question 
  \begin{parts}
    \item Show that if $(x,y)$ is a solution to $x^2-2y^2=1$ then so is $(3x+4y,2x+3y)$.
    \item This means that the matrix $A$ below carries given solutions to new solutions.
    \[A=\begin{bmatrix}3&4\\2&3\end{bmatrix}
    \]
    Let $\mathbf{x_0}=\begin{bmatrix}1\\0\end{bmatrix}$. Find the first few powers $A\mathbf{x_0}$, $A^2\mathbf{x_0}$, $A^3\mathbf{x_0}$.
    \item If you are familiar with matrix diagonalization, use it to find a general formula for the power $A^n\mathbf{x_0}$.
  \end{parts}
  \newpage
  \question 
  \begin{parts}
    \item Consider the first nontrivial solution $(3,2)$ to the equation $x^2-2y^2=1$. Calculate and simplify $(3+2\sqrt{2})(3+2\sqrt{2})$.
    \item Suppose that $(x,y)$ is a solution to the equation $x^2-2y^2=1$. Calculate and simplify $(x+y\sqrt{2})(3+2\sqrt{2})$. Show that the resulting coefficients give a solution to $x^2-2y^2=1$.
  \end{parts}
\end{questions}
\end{document}
