\documentclass[12pt]{exam}

\newcommand{\docname}{Sums of squares activity}

% \printanswers
\lhead[Math 404/504]{Page \thepage\ of \numpages}
\rhead{\docname}
\cfoot{}
\headrule
\usepackage{mathpazo,amsmath,amssymb}
\newcommand{\set}[1]{\left\{\,#1\,\right\}}
\newcommand{\N}{\mathbb N}
\newcommand{\Z}{\mathbb Z}
\newcommand{\Q}{\mathbb Q}
\newcommand{\R}{\mathbb R}
\renewcommand{\labelenumi}{(\alph{enumi})}
\renewcommand{\labelitemi}{$\circ$}
\usepackage{tikz}

\usepackage{listings}
\lstset{basicstyle=\small,frame=l,xleftmargin=.5in}

\begin{document}
\begin{questions}
  \question We now know that $-1$ is a square in $\Z_p$ if and only if $p\equiv1\pmod{4}$. We now investigate how to find a square root of $-1$ when we know one exists.

  We know from Euler's lemma that $a^{p-1}\equiv 1\pmod{p}$. We furthermore know that means $a^{(p-1)/2}\equiv \pm1\pmod{p}$. Finally, that means $a^{(p-1)/4}$ has a 50\% chance of being a square root of $-1$.

  Use our \texttt{rabinmillertable} function, and particularly the $(p-1)/4$ column, to find the square roots of $-1$ for each of the following primes. For very large primes, you may wish to simply try several values of $a^{(p-1)/4}$ rather than produce the entire table.

  \newpage
  \question 
  
\end{questions}
\end{document}
