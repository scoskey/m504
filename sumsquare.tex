\documentclass[12pt]{exam}

\newcommand{\docname}{Sums of squares activity}

% \printanswers
\lhead[Math 404/504]{Page \thepage\ of \numpages}
\rhead{\docname}
\cfoot{}
\headrule
\usepackage{mathpazo,amsmath,amssymb}
\newcommand{\set}[1]{\left\{\,#1\,\right\}}
\newcommand{\N}{\mathbb N}
\newcommand{\Z}{\mathbb Z}
\newcommand{\Q}{\mathbb Q}
\newcommand{\R}{\mathbb R}
\renewcommand{\labelenumi}{(\alph{enumi})}
\renewcommand{\labelitemi}{$\circ$}
\usepackage{tikz}

\usepackage{listings}
\lstset{basicstyle=\small,frame=l,xleftmargin=.5in}

\begin{document}
\begin{questions}
  \question We now know that $-1$ is a square in $\Z_p$ if and only if $p\equiv1\pmod{4}$. We now investigate how to find a square root of $-1$ when we know one exists.

  We know from Euler's lemma that $a^{p-1}\equiv 1\pmod{p}$. We furthermore know that means $a^{(p-1)/2}\equiv \pm1\pmod{p}$. Finally, that means $a^{(p-1)/4}$ has a 50\% chance of being a square root of $-1$.

  We may use our \texttt{rabinmillertable} function to find square roots of $-1$, by focusing on the $(p-1)/4$ column.
  
  Find the square roots of $-1$ for each of the following primes congruent to $1$ modulo $4$.
  \begin{parts}
    \item $p=29$
    \vspace\fill
    \item $p=101$
    \vspace\fill
    \item $p=592453$
    
    (Don't print the entire Rabin--Miller table. Simply try some values of $a^{(p-1)/4}$)
    \vspace\fill
  \end{parts}
  \newpage
  \question We now investigate the goal of writing $p$ as a sum of squares $p=a^2+b^2$. To illustrate the procedure, we begin with $p=29$.
  \begin{parts}
  	\item To begin, write $a^2+b^2=Mp$, where $a=$ one of your answers from 1(a), $b=1$ and $M$ is any integer.
		\vspace\fill
		\item We have now written $Mp$ as a sum of squares. Our next goal is to use this to write $mp$ as a sum of squares for some $m<M$. To begin, find numbers $u,v$ such that $u\equiv a\pmod{M}$, $v\equiv b\pmod{M}$, and $|u|,|v|\leq M/2$. (Thus $u,v$ can be negative if necessary.)
		\vspace\fill
		\item Find $m$ such that $u^2+v^2=Mm$. (Note that $m$ can be $1$.)
		\vspace\fill
		\item We now have $(u^2+v^2)(a^2+b^2)=M^2mp$. Use the identity below to write $M^2mp$ as a sum of squares.
		\[(u^2+v^2)(A^2+b^2)=(ua+vb)^2+(va-ub)^2
		\]
		\vspace\fill
		\item Divide both sides by $M^2$ to write $mp$ as a sum of squares.
		\vspace\fill
		\item If $m=1$ we are done. If $m>1$, repeat steps (b)--(e) to shrink again!
	\end{parts}
	\newpage
	\question Now that we have seen the method, try it again for $p=1973$.
	\begin{parts}
		\item Find a square root of $-1$ modulo $p$ and call it $a$. Let $b=1$ and write $a^2+b^2=Mp$.
		\item Find $u,v\equiv a,b\pmod{M}$ with $|u|,|v|\leq M/2$.
		\item Find $m$ such that $u^2+v^2=Mm$.
		\item Use $(ua+vb)^2+(va-ub)^2=M^2mp$ to write $M^2mp$ as a sum of squares.
		\item Divide both sides by $M^2$ to write $mp$ as a sum of squares.
		\item Repeat steps (b)--(e) until done.
	\end{parts}
\end{questions}
\end{document}
