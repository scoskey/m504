\documentclass[12pt]{exam}

\newcommand{\docname}{Modular powers activity}

% \printanswers
\lhead[Math 404/504]{Page \thepage\ of \numpages}
\rhead{\docname}
\cfoot{}
\headrule
\usepackage{mathpazo,amsmath,amssymb}
\newcommand{\set}[1]{\left\{\,#1\,\right\}}
\newcommand{\N}{\mathbb N}
\newcommand{\Z}{\mathbb Z}
\newcommand{\Q}{\mathbb Q}
\newcommand{\R}{\mathbb R}
\renewcommand{\labelenumi}{(\alph{enumi})}
\renewcommand{\labelitemi}{$\circ$}
\usepackage{tikz}

\usepackage{listings}
\lstset{basicstyle=\small,frame=l,xleftmargin=.5in}

\begin{document}
\begin{questions}
  \question In the last activity we arrived at Fermat's little theorem, which states that if $p$ is prime then $a^p\equiv1\pmod{p}$.
  
  The asme argument may be used to show Euler's thoerem: if $n$ is any integer and $\gcd(a,n)=1$, then $a^{\phi(n)}\equiv1\pmod{p}$, where $\phi(n)$ is the number of invertible elements of $\Z_n$.
  
  This leads us to the question, what is $\phi(n)$ equal to? To investigate this, start by collecting some data for small $n$. You may use the code below to show and count the invertible elements of $\Z_n$.
  
  \begin{lstlisting}
def euler(n):
  from math import gcd
  list = ''
  count = 0
  for i in range(n):
    d = gcd(i,n);
    if d==1:
      list += f'{i:3}'
      count += 1
  print (list)
  return count
  \end{lstlisting}
  
  \begin{center}
    \begin{tikzpicture}
      \draw(-2,0)--(2,0);
      \draw(0,1)--(0,-10);
      \node at (-1,.5) {$n$};
      \node at (1,.5) {$\phi(n)$};
    \end{tikzpicture}
  \end{center}
  What patterns do you observe?
  \newpage
  \question The data looks simpler if we focus on simpler numbers. Here we look just at prime numbers and their powers.
  \begin{center}
    \begin{tikzpicture}
      \draw(-2,0)--(2,0);
      \draw(0,1)--(0,-10);
      \node at (-1,.5) {$2^k$};
      \node at (1,.5) {$\phi(2^k)$};
    \end{tikzpicture}
    \qquad
    \begin{tikzpicture}
      \draw(-2,0)--(2,0);
      \draw(0,1)--(0,-10);
      \node at (-1,.5) {$3^k$};
      \node at (1,.5) {$\phi(3^k)$};
    \end{tikzpicture}
    \qquad
    \begin{tikzpicture}
      \draw(-2,0)--(2,0);
      \draw(0,1)--(0,-10);
      \node at (-1,.5) {$5^k$};
      \node at (1,.5) {$\phi(5^k)$};
    \end{tikzpicture}
  \end{center}
  What patterns do you observe? Can you explain why the pattern is always true?
  \newpage
  \question With the goal of finding $\phi(n)$ for more general numbers $n$, we next look at products of two different prime powers. For each table entry, multiply the row and column labels and then find $\phi$ of the product.
  \begin{center}
    \begin{tikzpicture}
      \draw(-1,0)--(6,0);
      \draw(0,1)--(0,-6);
      \foreach \i in {0,...,5} {
        \node at (\i+.5,.5) {$\pgfmathparse{2^\i}\pgfmathprintnumber[precision=1]{\pgfmathresult}$};
      }
      \foreach \i in {0,...,5} {
        \node at (-.5,-\i-.5) {$\pgfmathparse{3^\i}\pgfmathprintnumber[precision=1]{\pgfmathresult}$};
      }
    \end{tikzpicture}
  \end{center}
  What patterns do you observe?
  \newpage
  \question The pattern on the previous page generalizes to the following elegant statement: if $\gcd(m,n)=1$, then $\phi(mn)=\phi(m)\phi(n)$. This is called the Chinese remainder theorem.
  \begin{parts}
    \part The CRT says that there are the same number of invertible elments $a\in\Z_{mn}$ as there are pairs of invertible elements $(b,c)\in\Z_m\times\Z_n$.
    
    Explain why given an invertible element $a\in\Z_{mn}$, the pair $(a\pmod{m},a\pmod{n})\in\Z_m\times\Z_n$ are both invertible.
    \vspace\fill
    \part We now only need to show that given a pair of invertible elements $(b,c)\in\Z_m\times\Z_n$, there is one and only one solution in $\Z_{mn}$ to the system of equations
    \[\set{x\equiv b\bmod{m},\ x\equiv{c}\bmod{n}}
    \]
    Practice solving such a system using specific numbers:
    \[\set{x\equiv 8\bmod{11},\ x\equiv{3}\bmod{19}}
    \]
    \vspace\fill
    \part Describe a general method for solving the system of equations
    \[\set{x\equiv b\bmod{m},\ x\equiv{c}\bmod{n}}
    \]
    \vspace\fill
  \end{parts}
\end{questions}
\end{document}


