\documentclass[12pt]{exam}

\newcommand{\docname}{Perfect numbers activity}

% \printanswers
\lhead[Math 404/504]{Page \thepage\ of \numpages}
\rhead{\docname}
\cfoot{}
\headrule
\usepackage{mathpazo,amsmath,amssymb}
\newcommand{\set}[1]{\left\{\,#1\,\right\}}
\newcommand{\N}{\mathbb N}
\newcommand{\Z}{\mathbb Z}
\newcommand{\Q}{\mathbb Q}
\newcommand{\R}{\mathbb R}
\renewcommand{\labelenumi}{(\alph{enumi})}
\renewcommand{\labelitemi}{$\circ$}
\usepackage{tikz}

\usepackage{listings}
\lstset{basicstyle=\small,frame=l,xleftmargin=.5in}

\begin{document}
\begin{questions}
  \question A perfect number is one that is equal to the sum of its divisors (not including itself). For example, 6 is perfect because its divisors are $\set{1,2,3,6}$ and the sum of the proper divisors is $1+2+3=6$. We can use code such as the following to find the first few perfect numbers.
  
  \begin{lstlisting}
def divisorsum(n):
    sum = 0;
    for i in range(1,n):
        if n%i==0:
            sum += i
    return sum

def findperfect(n):
    output = []
    for i in range(1,n):
        if divisorsum(i)==i:
            output.append(i)
    return output
  \end{lstlisting}
  
  Try running the code, and make observations. Perfect numbers are rare, so you may not get more than a few!
  \newpage
  \question It may not be apparent
\end{questions}
\end{document}


