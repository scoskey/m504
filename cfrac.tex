\documentclass[12pt]{exam}

\newcommand{\docname}{Continued fractions activity}

% \printanswers
\lhead[Math 404/504]{Page \thepage\ of \numpages}
\rhead{\docname}
\cfoot{}
\headrule
\usepackage{mathpazo,amsmath,amssymb}
\newcommand{\set}[1]{\left\{\,#1\,\right\}}
\newcommand{\N}{\mathbb N}
\newcommand{\Z}{\mathbb Z}
\newcommand{\Q}{\mathbb Q}
\newcommand{\R}{\mathbb R}
\renewcommand{\labelenumi}{(\alph{enumi})}
\renewcommand{\labelitemi}{$\circ$}
\usepackage{tikz}

\usepackage{listings}
\lstset{basicstyle=\small,frame=l,xleftmargin=.5in}

\begin{document}
\begin{questions}
  \question Given any irrational number $x$, we can find its continued fraction as follows. Initially set $x_0=x$ and $a_0=\lfloor x_0\rfloor$. Then if $x_n,a_n$ are defined set $x_{n+1}=1/(x_n-a_n)$ and $a_{n+1}=\lfloor x_n\rfloor$.
  
  Use this method to find the continued fractions for $\sqrt{3}$ and $\sqrt{5}$ by hand.
  \newpage
  \question Given any $x$, we can use a computer to find the first $b$ terms of the continued fraction.
  \begin{parts}
    \item Use our \texttt{cfrac(x,b)} function to find the first 15 terms of the continued fractions for each of the following numbers.
    \[\sqrt{2}, \sqrt{3}, \sqrt{5}, \sqrt{6}, \sqrt{7}, \sqrt[3]{2}, \pi
    \]
    You may wish to: \texttt{from sympy import sqrt}, and \texttt{from sympy import pi}. Do you notice any patterns?
    \item Given a continued fraction, we can simplify it to a sequence of ordinary fractions called convergents. Use our \texttt{convergents(a)} function to find the first 5 convergents for each of the numbers in the previous part. In each case, how close is the 5th convergent to the original number?
  \end{parts}
  \newpage
  \question Some continued fractions are periodic, meaning that they eventually repeat. Given a periodic continued fraction, it is possible to find the irrational number it came from using the repetition.
  
  For instance, the number $x$ defined by the continued fraction $[2,2,2,\ldots]$ satisfies $x-2=1/x$. Therefore $x(x-2)=1$ and so $x$ must be a solution to $x^2-2x-1=0$.
  
  Use this method to find the irrational number $x$ defined by the continued fractions below.
  \begin{parts}
    \item $[1,1,1,1,1,1,\ldots]$
    \item $[1,3,1,3,1,3,\ldots]$
    \item $[1,1,2,3,2,3,2,3,\ldots]$
  \end{parts}
\end{questions}
\end{document}
