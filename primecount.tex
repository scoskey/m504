\documentclass[12pt]{exam}

\newcommand{\docname}{Primes counting activity}

% \printanswers
\lhead[Math 404/504]{Page \thepage\ of \numpages}
\rhead{\docname}
\cfoot{}
\headrule
\usepackage{mathpazo,amsmath,amssymb}
\newcommand{\set}[1]{\left\{\,#1\,\right\}}
\newcommand{\N}{\mathbb N}
\newcommand{\Z}{\mathbb Z}
\newcommand{\Q}{\mathbb Q}
\newcommand{\R}{\mathbb R}
\renewcommand{\labelenumi}{(\alph{enumi})}
\renewcommand{\labelitemi}{$\circ$}
\usepackage{tikz}

\usepackage{listings}
\lstset{basicstyle=\small,frame=l,xleftmargin=.5in}

\begin{document}
\begin{questions}
  \question We have seen there are infinitely many primes. But how sparsely are they distributed in the natural numbers? We let $\pi(n)$ denote the number of primes $<n$. We can use code such as the following to produce the first few values of $\pi(n)$.
  
  \begin{lstlisting}
def countprimes(n):
    from sympy import isprime
    pivals = [0]
    for i in range(1,n):
        if isprime(i):
            pivals.append( pivals[i-1]+1 )
        else:
            pivals.append( pivals[i-1] )
    return pivals
  \end{lstlisting}
  
  Try running the code on $n=10$ and $n=100$. What patterns do you observe?
  \newpage
  \question Now we can use Desmos to plot the function $\pi(n)$. For example, to plot $\pi(0),\ldots,\pi(99)$, first enter the following into a box:
  \[x_1=[0,\ldots,99]
  \]
  Next copy the output of countprimes(100) to your clipboard and enter the following into a box:
  \[y_1=\emph{paste here}
  \]
  Finally enter the following into a box:
  \[(x_1,y_1)
  \]
  My plot looks like the one shown below (though I did adjust the style in the graph settings). If you have trouble getting here, ask for help!
  \begin{figure}[h]
    \centering
    \includegraphics[width=.7\textwidth]{desmos}
  \end{figure}
  
  Try to find a function that ``fits'' the data. For example you might enter the following into a box:
  \[\sqrt{x}
  \]
  You will see that $y=\sqrt{x}$ isn't a good fit for $\pi(x)$. Can you find a function that is a better fit?
 
  Consider adding more than 100 data points for more accurate conclusions.
\end{questions}
\end{document}


