\documentclass[12pt]{exam}

\newcommand{\docname}{Quadratic residues activity}

% \printanswers
\lhead[Math 404/504]{Page \thepage\ of \numpages}
\rhead{\docname}
\cfoot{}
\headrule
\usepackage{mathpazo,amsmath,amssymb}
\newcommand{\set}[1]{\left\{\,#1\,\right\}}
\newcommand{\N}{\mathbb N}
\newcommand{\Z}{\mathbb Z}
\newcommand{\Q}{\mathbb Q}
\newcommand{\R}{\mathbb R}
\renewcommand{\labelenumi}{(\alph{enumi})}
\renewcommand{\labelitemi}{$\circ$}
\usepackage{tikz}

\usepackage{listings}
\lstset{basicstyle=\small,frame=l,xleftmargin=.5in}

\begin{document}
\begin{questions}
  \question A nonzero number $a$ is a \emph{quadratic residue} or simply a \emph{square} modulo $p$ if there exists $b$ such that $a\equiv b^2\pmod{p}$. The following code may be used to determine which numbers are squares modulo $p$.
  \begin{lstlisting}
def squares(p):
    output = set()
    for b in range(1,p):
        output.add( pow(b,2,p) )
    output = list(output)
    output.sort()
    return( output )
  \end{lstlisting}
  Try running \texttt{squares} on several primes $p$. What observations do you make?
  \begin{parts}
    \part How many square numbers are there modulo $p$?
    \part Can you explain why the squares $i^2$ and $(p-i)^2$ are the same?
    \part Can you explain why no other squares are the same?
    \part What happens when you multiply two squares together modulo $p$? Two non-squares?
  \end{parts}
  \newpage
  \question Use the \texttt{rabinmillertable} function to look at the column of values of $a^{(p-1)/2}$. We have previously observed that this column consists entirely of $1$'s and $-1$'s. But how can you tell whether it will be $1$ or $-1$?
  \begin{parts}
    \part Can you explain why the only possible values for $a^{(p-1)/2}$ modulo $p$ are $1$ and $-1$? [Hints: Think about the $F\ell T$, and the number of possible roots of a quadratic.]
    \part Prove that if $a$ is a square, then $a^{(p-1)/2}$ is equivalent to $1$ modulo $p$.
    \part Can you explain why if $a$ is a nonsquare, then $a^{(p-1)/2}$ must be equivalent to $-1$ modulo $p$?
  \end{parts}
  \newpage
  \question The Legendre symbol $(\frac ap)$ is defined to be $1$ if $a$ is a square modulo $p$ and to be $-1$ if $a$ is a nonsquare modulo $p$. The following code may be used to calculate values of the Legendre symbol using this definition.
  \begin{lstlisting}
def legendre(a,p):
    if a%p ==0:
        return 0
    elif a%p in squares(p):
        return 1
    else:
        return -1
  \end{lstlisting}
  In this exercise we investigate the values of $(\frac {-1}{p})$ for primes $p$.
\end{questions}
\end{document}


