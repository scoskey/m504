\documentclass[12pt]{exam}

\newcommand{\docname}{Square roots activity}

% \printanswers
\lhead[Math 404/504]{Page \thepage\ of \numpages}
\rhead{\docname}
\cfoot{}
\headrule
\usepackage{mathpazo,amsmath,amssymb}
\newcommand{\set}[1]{\left\{\,#1\,\right\}}
\newcommand{\N}{\mathbb N}
\newcommand{\Z}{\mathbb Z}
\newcommand{\Q}{\mathbb Q}
\newcommand{\R}{\mathbb R}
\renewcommand{\labelenumi}{(\alph{enumi})}
\renewcommand{\labelitemi}{$\circ$}
\usepackage{tikz}

\usepackage{listings}
\lstset{basicstyle=\small,frame=l,xleftmargin=.5in}

\begin{document}
\begin{questions}
  \question We know from quadratic reciprocity how to determine when a number $a$ has a square root modulo a prime $p$. We now begin to investigate how to find a square root of $a$ when we know one exists.

  First suppose that $p\cong3\pmod{4}$.
  \begin{parts}
    \part Show you can write $p-1=2s$ where $s$ is odd.
    \vspace{1in}
    \part Show that if $a$ is a square modulo $p$, then $x=a^{\frac{s+1}{2}}$ is a solution to $x^2\equiv a\pmod{p}$. [Hint: Remember Euler's criterion.]
    \vspace\fill
    \part Try your answer on $p=$ and $a=$.
    \vspace\fill
    \part Try your answer on $p=$ and $a=$.
    \vspace\fill
  \end{parts}
  \newpage
  \question Now we look at the case when $p\equiv1\pmod{4}$. Since this case is more difficult than before, we next look just at the subcase when $p\equiv5\pmod{8}$.
  \begin{parts}
    \part Show you can write $p-1=4s$ where $s$ is odd.
    \vspace{1in}
    \part Show that if $a$ is a square modulo $p$, then $x=a^{\frac{s+1}{2}}$ is a solution to $x^2\equiv\pm a\pmod{p}$.
    \vspace\fill
    \part If $x^2\equiv a\pmod{p}$ you are done. Otherwise to complete our solution, we seek a square root $z$ of $-1$ modulo $p$. Let $z=2^s$ and show that $z^2\equiv-1\pmod{p}$.
    \vspace\fill
    \part Let $x_{\rm new}=xz$ and confirm that $x_{\rm new}$ is a square root of $a$ modulo $p$.
    \vspace\fill
    \part Try your method on $p=$ and $a=$.
    \vspace\fill
    \part Try your method on $p=$ and $a=$.
    \vspace\fill
  \end{parts}
  \newpage
  \question Finally we will show part of the general method. For this problem let $p=41$, and follow along to find a square root of $a=5$ modulo $p$.
  \begin{parts}
    \part Write $p-1=2^rs$ where $s$ is odd.
    \vspace{.5in}
    \part Show that if $a$ is a square modulo $p$, then $x=a^{\frac{s+1}{2}}\pmod{p}$ is a solution to $x^2\equiv a^sa\pmod{p}$.
    \vspace\fill
    \part To complete our solution, we seek a square root $z$ of $(a^s)^{-1}$ modulo $p$. To begin, find the value of $(a^s)^{-1}\pmod{p}$.
    \vspace\fill
    \part This is a side step. Find any quadratic nonresidue $N$ and calculate the value of $M=N^s$.
    \vspace\fill
    \part Some group theory tells us that $(a^s)^{-1}$ will equal an even power of $M$. Try the powers $2,4,6,\ldots$ until you find an even power $2k$ such that $M^{2k}\equiv(a^s)^{-1}\pmod{p}$.
    \vspace\fill
    \part Let $z=M^k\pmod{p}$. Find $x_{\rm new}=xz$ and confirm that $x_{\rm new}$ is a square root of $a$ modulo $p$.
    \vspace\fill
  \end{parts}
  We should note that there is a more efficient way to find the correct even power of $M$, but we will leave that alone for today!
\end{questions}
\end{document}

