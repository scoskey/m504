\documentclass[12pt]{exam}

\newcommand{\docname}{ElGamal activity}

% \printanswers
\lhead[Math 404/504]{Page \thepage\ of \numpages}
\rhead{\docname}
\cfoot{}
\headrule
\usepackage{mathpazo,amsmath,amssymb}
\newcommand{\set}[1]{\left\{\,#1\,\right\}}
\newcommand{\N}{\mathbb N}
\newcommand{\Z}{\mathbb Z}
\newcommand{\Q}{\mathbb Q}
\newcommand{\R}{\mathbb R}
\renewcommand{\labelenumi}{(\alph{enumi})}
\renewcommand{\labelitemi}{$\circ$}
\usepackage{tikz}

\usepackage{listings}
\lstset{basicstyle=\small,frame=l,xleftmargin=.5in}

\begin{document}
\begin{questions}
  \question If $g$ is a primitive root modulo $p$, exponentiation $g^x\pmod{p}$ can be used to make a ``shared secret'' between two people. If I choose an exponent $x$ and send you $g^x$, and you choose an exponent $y$ and send me $g^y$, then we can both find $s=g^{xy}$.
  
  Work with the prime modulus $p=41$.
  \begin{parts}
    \item Use the \texttt{powertable} function to find the smallest primitive root $g$ for $\Z_{41}$.
    \vspace\fill
    \item Choose any exponent $x\in2,\ldots p-2$ and \textbf{keep it a secret}. Calculate $g^x\pmod{p}$ and tell it to the group. This is your ``public key''.
    \vspace\fill
    \item Pick any member of your group to work with first. Write down their public key and call it $g^y$ (even though you don't know $y$). Then calculate $s=(g^y)^x\pmod{p}$. This is a ``shared secret'' between the two of you! Use direct messaging to confirm you have the same value for $s$.
    \vspace\fill
    \item Repeat the above so that every pair in the group has a shared secret between them. For future use, label each shared secret with the name of the other person involved.
    \vspace\fill
  \end{parts}
  \newpage
  \question A shared secret $s$ may be used to encrypt a message $m$ simply by multiplying the message and secret together: $e=ms$ modulo $p$.
  \begin{parts}
    \item Continuing from the previous page, think up a short message to send (you can only send one letter at a time). Encode your message using the table in Chapter 18, or our \texttt{text2ints} function. You will have a list of $2$-digit integers $[m_i]$.
    \vspace\fill
    \item Pick someone in the group to send your message to, and let $s$ be your shared secret with that person. For each block $m_i$, calculate $e_i=m_is\pmod{p}$. Give the list of encrypted blocks $[e_i]$ to the other person.
    \vspace\fill
    \item You will also receive message blocks $[e_i]$ from the other person. With $s$ still the shared secret with that person, use our \texttt{inverse} function to calculate $s^{-1}\pmod{p}$ and then calculate each $m_i=e_is^{-1}\pmod{p}$.
    \vspace\fill
    \item Decode the message using the table or our \texttt{ints2text} function. Confirm that you received the message properly.
    \vspace\fill
    \item Repeat the above so that everyone in the group has a chance to send and receive a message.
    \vspace\fill
  \end{parts}
  \newpage
  \question (See Silverman, exercise 29.7.) You receive a series of message blocks that were encrypted using the prime $p=380803$, the primitive root $g=2$, and your public key. Each block was encrypted with a different secret exponent for security:
  \begin{align*}
    g^{y_1}=61745,&\quad e_1=206881\\
    g^{y_2}=255836,&\quad e_2=314674\\
    g^{y_3}=108147,&\quad e_3=350768
  \end{align*}
  \begin{parts}
    \item Your own secret exponent is $x=278374$. Find the three shared secrets $s_1,s_2,s_3$.
    \vspace\fill
    \item Find the inverse of each shared secret $(s_1)^{-1},(s_2)^{-2},(s_3)^{-3}$, all modulo $p$.
    \vspace\fill
    \item Find the three message blocks $m_1=e_1(s_1)^{-1}$, $m_2=e_2(s_2)^{-1}$, $m_3=e_3(s_3)^{-1}$, all modulo $p$.
    \vspace\fill
    \item Decode the message $[m_1,m_2,m_3]$:
    \vspace\fill
  \end{parts}
\end{questions}
\end{document}
